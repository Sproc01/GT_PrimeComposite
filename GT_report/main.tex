\documentclass[conference]{IEEEtran}
\IEEEoverridecommandlockouts
% The preceding line is only needed to identify funding in the first footnote. If that is unneeded, please comment it out.
\usepackage{cite}
\usepackage{amsmath,amssymb,amsfonts}
\usepackage{algorithmic}
\usepackage{graphicx}
\usepackage{textcomp}
\usepackage[hidelinks]{hyperref}
\newcommand*{\Comb}[2]{{}_{#1}C_{#2}}%

\usepackage{tabularx,ragged2e}
\newcolumntype{C}{>{\Centering\arraybackslash}X} % centered "X" column
% \newcolumntype{b}{X}
\newcolumntype{s}{>{\hsize=.5\hsize}X}

\usepackage{xcolor}
\def\BibTeX{{\rm B\kern-.05em{\sc i\kern-.025em b}\kern-.08em
    T\kern-.1667em\lower.7ex\hbox{E}\kern-.125emX}}
\begin{document}

\title{Primi Composti: a Tabletop Game Analysis}

\author{
\IEEEauthorblockN{1\textsuperscript{st} Lorenzo Serafini}
\IEEEauthorblockA{
\textit{University Of Padua}\\
Padua, Italy \\
lorenzo.serafini.1@studenti.unipd.it}
\and
\IEEEauthorblockN{2\textsuperscript{nd} Michele Sprocatti}
\IEEEauthorblockA{
\textit{University Of Padua}\\
Padua, Italy \\
michele.sprocatti@studenti.unipd.it \\ https://orcid.org/0009-0005-7886-441X}
\and
\IEEEauthorblockN{3\textsuperscript{rd} Riccardo Zuech}
\IEEEauthorblockA{
\textit{University Of Padua}\\
Padua, Italy \\
riccardo.zuech@studenti.unipd.it}
}
\maketitle

\begin{abstract}
	This project has the objective of providing a game theoretic analysis of a real-world scenario, that is the tabletop game Primi Composti. In particular, we are interested in checking whether there is an inherent advantage in going first or second by considering all combinations of different practical strategies from a set we define.
	This project tackles some of the challenges that arise when applying game theoretic tools to real-world practical scenarios, like dealing with the random assignments of cards to the players and the high computational costs of standard best-response strategies. So in order to deal with these aspects, we have tried to find a balance over the number of samples and the number of strategies in order to have statistical significance and also to not have too much computational time required to run all the computations.
    After the simulations, some analyses are performed in order to understand if there are some advantages and if the game is fair enough for both players.
    Finally a minimization method is used to find the best binomial distribution that best fits the data resulting by the simulations.
\end{abstract}

\section{Introduction} \label{section:Introduction}% Riccardo
\section{Strategies} \label{section: Strategies}% Lorenzo
As already mentioned, finding strategies to obtain SPE equilibria is a computationally expensive task that is not feasible for our game.
However, we still want to test the evolution of the game under reasonable strategies, in order to see the distribution of wins and check whether any of them outperform the others.

In particular, we implemented the following strategies:
\begin{itemize}
    \item random strategy: the player plays a random card from their hand;
    \item max strategy: the player plays the card with the maximum number written on it;
    \item min strategy: the player plays the card with the minimum number written on it;
    \item prime\_first strategy: the player plays at first only prime numbers cards, if any, chosen at random, otherwise he plays composite numbers card, again chosen in random order;
    \item composite\_first strategy: the player plays at first only composite numbers cards, if any, chosen at random, otherwise he plays prime numbers card, again chosen in random order;
    \item max\_val strategy: play the card that gives the highest amount of points, given the current set of visible cards on the table;
\end{itemize}

Additionally, as mentioned before, we can leverage the zero-sum characteristics of the game and take inspiration from the concepts of security strategies and minimax to define our own approximated versions.
The idea is essentially to see each turn and the following one as a static stage-like game; factoring in the zero-sum modelling of the game, this allows the player to choose at each turn a card to play as it would be dictated by the maximin and minimax paradigm in pure strategies. It is needless to say that this is a huge simplification: we are just considering the current turn and the opponent response as a static game, even though it is actually a Stackelberg relation and just a small component of the whole picture of the game.
Given this idea, we define the two following strategies:
\begin{itemize}
	\item security strategy: compute the utility matrix of all possible outcomes of the current turn and the next one as a static game (as described above), then choose the card that maximises the minimum amount of points achievable, i.e. the payoff in the worst case of opponent's play.
	\item minimax strategy: compute the utility matrix of all possible outcomes of the current turn and the next one as a static game, then choose the card that minimizes the maximum amount of points achievable, i.e. the minimum amount of points achievable by the player if it could perfectly predict its opponent's move. 
\end{itemize}

Building on this idea, one may be able to obtain better results by applying instead "best-response"-like strategies using backward induction on the small stage-like Stackelberg modelling of the current turn and the next one, however they would incur again in the practical infeasibility of simulating a significant number of games using such a strategy. Additionally, one may also be able to generalize those two strategies to the general case of more players by also building a belief system and keep the computational costs contained; however, this idea is not explored further since it goes out of the focus of our analysis, but it may prove interesting as future work.
\input{sections/Simulation2}
%\section{Simulation} \label{section:Simulation} %Michele
To understand if the first player has some moving advantage, we had to run several simulations by comparing the different strategies explained in section \ref{section: Strategies} in the two cases that we want to compare:
\begin{itemize}
    \item General case: the starting player is decided by the random cards that they got.
    \item Player 1 always starting: we forced the fact that the player 1 has the card corresponding to number two so he starts always the game.
\end{itemize}
In order to achieve this, firstly we generate the different possible configurations of cards by assigning each card uniformly at random between the two players. Then for each assignment, we run the game comparing every possible pair of strategies from the set explained in \ref{section: Strategies}.

\subsection{Results} \label{subsection:Results}
To have enough data to make statistical analyses but also to have limited computation time, we decided to do $10 000$ samples of possible assignments; then we simulate the different strategies.


By looking at figure \ref{fig:comparison_pl1}, that plots the win of player 1 when the two players play the same strategy, we can see that there is some small advantage for some strategies (random, max, min, prime, comp, max\_val) versus a great worsening for two particular strategies (security, minimax).

This is also reflected in the opposite plot, so the one for player 2 reported in figure \ref{fig:comparison_pl2}.

\begin{figure}
    \centering
    \includegraphics[width=0.9\linewidth]{img/comparison_winning_starts_general_pl1.png}
    \caption{Wins of player 1 in the two cases when the two players play the same strategy.}
    \label{fig:comparison_pl1}
\end{figure}

\begin{figure}
    \centering
    \includegraphics[width=0.9\linewidth]{img/comparison_winning_not_starts_general_pl2.png}
    \caption{Wins of player 2 in the two cases when the two players play the same strategy.}
    \label{fig:comparison_pl2}
\end{figure}

After these, we have computed the probability that player 1 wins the game for every possible pair of strategies. The probability is computed as follows:

\begin{equation} \label{eq:probability_cal}
    prob = \frac{number-of-wins}{number-of-total-matches}
\end{equation}

To visualize the results in these two cases, we decided to use the heat-maps; the plots are reported in figure \ref{fig:prob_general} and in figure \ref{fig:prob_starts}. 

From the plot in figure \ref{fig:prob_general} we can see that in the general case the probability of winning for player 1 is very high when these combinations arise:
\begin{enumerate}
    \item max\_val vs random
    \item security vs random
    \item minimax vs random
    \item max\_val vs min
    \item security vs min
    \item minimax vs min
    \item security vs prime first
    \item minimax vs prime first
\end{enumerate}
The first 3 were somehow obvious because the second player is choosing the card randomly without a strategy, so a good strategy for player 1 will very likely beat player 2. The other five results instead are more interesting. 
Since the probability of winning for player 2 is $1 - probability-winning-player-1$ we can see the dark part at the top right of the heat-map.

By looking at figure \ref{fig:prob_starts} we can see what we have pointed out before using figure \ref{fig:comparison_pl1} and \ref{fig:comparison_pl2} that is the fact that we have a significant drop in the chance of winning for the strategies in the bottom right part of the heat-map, but we have an increasing probability for all the other combinations.

To visualize in a better way this last observation we plot also the difference between the two sets of probabilities and we obtained the heat-map in \ref{fig:diff_prob}, using this plot we can see that we have a 5 \% increment in the top left but a 25 \% decrease in the bottom right. So given these values we can say that maybe the player that does not start has an advantage because in certain pairs of strategies the probability of the starting player increases by a small percentage but it decreases a lot for other strategies so the not-starting player has in general a greater chance to win.

\begin{figure}
    \centering
    \includegraphics[width=0.9\linewidth]{img/prob_winning_general.png}
    \caption{Probability of winning for player 1 in general case (strategy of player 1 on the rows).}
    \label{fig:prob_general}
\end{figure}

\begin{figure}
    \centering
    \includegraphics[width=0.9\linewidth]{img/prob_winning_starts.png}
    \caption{Probability of winning for player 1 in the case where he always starts (strategy of player 1 on the rows).}
    \label{fig:prob_starts}
\end{figure}

\begin{figure}
    \centering
    \includegraphics[width=0.9\linewidth]{img/diff_prob.png}
    \caption{Difference between probability of winning for player 1 when it starts and the general case.}
    \label{fig:diff_prob}
\end{figure}

For all possible simulations that we have run, we also collected the different scores of each match. These results are used in order to plot a box plot for each pair of strategies in the two cases that we have analyzed. These can be seen in figure \ref{fig:box_general} and figure \ref{fig:box_starts}.
From the comparison between these plots, we can see again the bottom left part where we have a worsening of the scores of player 1 from general to the case where it starts confirming again what we have said before. 
Also, we can see that, in general, we have very small boxes but very large whiskers; this means that we have a large variance of the scores but also that in the 2\textsuperscript{nd} and 3\textsuperscript{rd} quantiles we have data that is very concentrated. We can also see that there are a lot of outliers pretty much everywhere. In particular, we can see that when player 1 starts, the number of outliers increases a lot for both players.

\begin{figure}
    \centering
    \includegraphics[width=1\linewidth]{img/box_plot_general.png}
    \caption{Plot that reports the box plots of the different score for each pair of strategies in the general case.}
    \label{fig:box_general}
\end{figure}

\begin{figure}
    \centering
    \includegraphics[width=1\linewidth]{img/box_plot_starts.png}
    \caption{Plot that reports the box plots of the different score for each pair of strategies when player 1 starts.}
    \label{fig:box_starts}
\end{figure}

At the end of the code, there is also a small part that uses an optimization method to find the binomial distribution that best fits the data in the two cases. The distributions found are very similar to the ones computed before using the equation \ref{eq:probability_cal}.
\section{Conclusions}
% I do not know

\section*{Link}
The code is available at \href{https://github.com/Sproc01/GT_PrimeComposite.git}{\textit{Repository}}

\section*{Acknowledgment}
Luciano Murrone that presented to us this game.

\section*{References}
\begin{itemize}
    \item L. Badia, T. Marchioro, Game theory - A handbook of problems and exercises, Esculapio, 2022
    \item S. Tadelis, Game Theory: an introduction. Princeton, 2013.
\end{itemize}


\end{document}

