\section{Conclusions} \label{section:Conclusions}
Our analysis indicates that the game, in its general form, exhibits a balanced design. The observed winning probabilities for the majority of strategies converge towards a 50/50 distribution, as you can see in figure \ref{fig:prob_general}, suggesting a fair game. While certain strategies deviate from this balance, leading to unbalanced probabilities, those instances are confined to a narrow subset of strategies, where one player plays an advanced strategy while the other plays a dumb strategy, like the random one.

To investigate potential first-mover or second-mover advantages, we analyzed the game dynamic with a fixed starting player. Our findings suggest that the game is balanced thanks to the cards being assigned randomly at the beginning, but if we fix the starting player, there is a first-mover advantage for simple strategies. However, the situation flips when advanced strategies come into play: these are very likely to result in a win for the player going second if its opponent always goes first.

Finally, by analyzing the game outcomes and score distributions, we have seen that advanced strategies inspired by game-theoretic paradigms significantly outperform simple ones, additionally achieving more consistent score distributions, and thus performances, compared to them.