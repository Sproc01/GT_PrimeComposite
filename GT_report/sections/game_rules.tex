\subsection{Game's rules}

In this work we focus on the 2 players scenario, where the rules are as follows.

There are 24 different cards whose value vary between 2 and 25, for a total of 9 primes and 15 composites in play.
The 24 cards get distributed at random to the players, building 2 hands of 12 cards; the player that received the 2 gets to play first.

The game is turn-based: at each turn a player has to play a card from its hand, receiving a certain amount of points that gets added to its score, and then pass the turn to its opponent, until they both run out of cards in the hand.

Each of the players has two piles of cards in front of himself, one for cards of prime values and one for composites; when a card is played, it gets placed on the top of its player's primes or composites pile depending on its value.

To conclude, Table \ref{tab:points} shows how the points are assigned to a certain play. At the end of the game, whoever obtained the highest score wins.

It is easy to see that, under this formulation, ties can happen. However, we consider a tie as an invalid result that needs to be repeated, in order to simplify the analysis.


\begin{table}[!ht]
	\setlength\extrarowheight{2pt} % for a bit of visual "breathing space"
	\begin{tabularx}{\linewidth}{|c|c|C|}
		\hline
		\textbf{Column 1} & \textbf{Column 2} & \textbf{Column 3}  \\
		\hline
		
		\hline
		\textbf{C} & 1  &  Whoever plays a composite card, whose value is not the result of any operation among the visible cards on the table, gets 1 point. \\
		\hline
		\textbf{P}& 2 &  Whoever plays a prime card, whose value is not the result of any operation among the visible cards on the table, gets 2 points. \\
		\hline
		\textbf{C = C \# C}& 3 &  Whoever plays a composite card, whose value can be seen the result of an operation between two composite cards visible on the table, gets 3 points. \\
		\hline
		\textbf{C = P \# C}& 4 & Whoever plays a composite card, whose value can be seen the result of an operation between a composite card and a prime card, both visible on the table, gets 4 points. \\
		\hline
		\textbf{P = C \# C}& 4 &  Whoever plays a prime card, whose value can be seen the result of an operation between two composite cards visible on the table, gets 4 points. \\
		\hline
		\textbf{P = P \# C}& 5 & Whoever plays a prime card, whose value can be seen the result of an operation between a composite card and a prime card, both visible on the table, gets 5 points \\
		\hline
		\textbf{P = P \# P}& 6 & Whoever plays a composite card, whose value can be seen the result of an operation between two prime cards visible on the table, gets 6 points. \\
		\hline
	\end{tabularx}
	\label{tab:points}
	\caption{Rules for assigning points to a player after it plays one of its cards. Symbol $\#$ indicates an arithmetic operation, while P stands for prime and C for composite.}
\end{table}