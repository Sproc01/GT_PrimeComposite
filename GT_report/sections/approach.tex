\subsection{Modelling approach}

It is straightforward to see that this game can essentially be modelled as a finite dynamic game.
In particular, it is interesting to notice how this game is of complete imperfect information, since the hands are given at random by Nature's choice.

However, it is worth observing how, after Nature's choice, the game becomes of complete perfect information, since the content of the other player's hand is determined given a player's hand, and thus common knowledge.
Additionally, it is worth pointing out how, after Nature's move, the game can be related to a Stackelberg game.
This is interesting because, after observing Nature's choice, each player can theoretically "solve" the game by always playing a best response through backward induction on the resulting decision tree.

Nevertheless, the backward induction approach, i.e. a strategy choice that would lead to a SPE, is infeasible to implement in practice, given that the resulting decision tree would have $12!\times12!$ possible different sequences of cards played (number of leaves), since the game is turn-based, one player will play only on odd turns and the other on even ones .
This major practical limitation hinders a rigorous analysis of the game, limiting us to only consider sub-optimal strategies due to the computational infeasibility of computing strategies that would yield subgame-perfect Nash equilibria.

Additionally, another major problem in analysing this game arises when considering the randomness of the hands: Nature has a total of \(\binom{24}{12}\) possible choices, meaning that the number of games to consider would be well beyond practical feasibility.
However, this problem is easily solvable since Nature's choice involves randomness, by just using an high enough number of trials to obtain results that are good approximations with high probability.

Another interesting aspect of this game is that it can be modelled also as a zero-sum game: it is straightforward to prove that if we do not only add the points of a play to the score of the relative player, but also subtract them from the score of its opponent, the final result, i.e. winning or losing the tabletop game, remains the same. More formally, it is easy to see how player 1 playing a card granting $x$ points gives partial utilities $u_1 = x$ and $u_2 = -x = -u_1$.
This aspect is particularly interesting when we consider possible computationally-practical strategies for our players, something that we will delve into in Section \ref{section: Strategies}.

Finally, we want to spend some words on potential error that would come up when trying to model this game.
First of all, one would be inclined to see it as a multistage game, however this would be an error since at each stage the hands of the players and the cards visible on the table depend on the outcomes of the previous stages, meaning that the stage games would not be independent of each other.
And on this note it is also important to recall that, given the impossibility to model the game as multistage, there is no way to obtain a SPE equilibrium aside for applying backward induction, which is infeasible in practice.
